\documentclass{article}
\author{Luís Spengler}
\title{Exercise List}
\date{\today}

\usepackage[margin=1cm]{geometry}
\usepackage{graphicx}
\usepackage{wrapfig}
\usepackage{caption}
\usepackage{subcaption}
\usepackage{titlesec}
\usepackage{amsmath}
\usepackage{physics}

\graphicspath{{./docs/}}

\titleformat{\section}
  {\sffamily}{\thesection}{1em}{}
\titleformat{\subsection}
  {\sffamily}{\thesubsection}{1em}{}
\renewcommand\thesubsection{\alph{subsection}}

\begin{document}
\maketitle
\sffamily

\section{Defina com suas palavras, o que é conversão eletromecânica da energia?} % This will be exercise n. 1
\section{Em quais as formas a energia na forma elétrica pode ser convertida?} % This will be exercise n. 2
\section{Explique com base na ilustração, quais fatores influenciam no valor da tensão induzida no condutor que corta o campo magnético.} % This will be exercise n. 3

% \includegraphics[width=\textwidth]{img1} % Some image you want to add
\section{Quais são as duas principais partes construtivas de qualquer máquina rotativa?} % This will be exercise n. 4
\section{Quais as formas mais comuns de se obter o campo magnético em uma máquina CC?} % This will be exercise n. 5

% Bellow you will add a group of two figures to be in the same row
%	\begin{figure}[h!]
%		\centering
%		\begin{subfigure}[R]{0.3\textwidth}
%			\centering
%			\includegraphics[width=0.7\textwidth]{img2}
%			\caption{Para questão 6}
%		\end{subfigure}
%		\hspace{2cm}
%		\begin{subfigure}[R]{0.3\textwidth}
%			\centering
%			\includegraphics[width=0.7\textwidth]{img3}
%			\caption{Para questão 7}
%		\end{subfigure}
%	\end{figure}
\section{A figura mostra uma bobina com \(N=200\) espiras. A resistência é \(R=100\Omega\); \(Ag=8cm*12cm; g= 6mm,\) sendo aplicada uma tensão \(V=150V_{DC}\), determine:} % This will be exercise n. 6
\subsection{Qual a intensidade inicial do campo no entreferro?} % This will be a subexercise.
\subsection{A força exercida sobre a parte móvel nesta distância de entreferro} % This will be a subexercise.
Após ligar a tensão elétrica, a parte móvel se movimentou, reduzindo o entreferro a \(2mm\). 
\subsection{A intensidade do campo e a força depois de ligado o circuito.} % This will be a subexercise.
\section{A figura mostra uma bobina com \(N=600\) espiras. A resistência é \(R=100\Omega\); \(Ag=10cm*10cm; g= 15mm,\) sendo aplicada uma tensão \(V=220V_{AC}, 60Hz\). Determine:} % This will be exercise n. 7
\subsection{A indutância, a reatância indutiva e a impedância da bobina:} % This will be a subexercise.
\subsection{Qual a intensidade inicial do campo no entreferro?} % This will be a subexercise.
\subsection{Força exercida sobre a parte móvel nesta distância de entreferro.} % This will be a subexercise.
\subsection{Após ligar a tensão elétrica, a parte móvel se movimentou, reduzindo o entreferro a \(3mm\).} % This will be a subexercise.
\subsection{Intensidade do campo e a força depois de ligado o circuito.} % This will be a subexercise.

\end{document}
