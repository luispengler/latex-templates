\documentclass{article}
\author{Luís Spengler}
\title{Lista de Exercícios - Conversão Eletromecânica}
\date{\today}

\usepackage[margin=1cm]{geometry}
\usepackage{amsmath}
\usepackage{physics}
\usepackage{titlesec}
\usepackage{graphicx}
\usepackage{wrapfig}
\usepackage{caption}
\usepackage{subcaption}

\graphicspath{{./docs/}}

\titleformat{\section}
  {\sffamily}{\thesection}{1em}{}
\titleformat{\subsection}
  {\sffamily}{\thesubsection}{1em}{}
\renewcommand\thesubsection{\alph{subsection}} % Accomplishes task 1

\begin{document}
\maketitle
\sffamily

\section{Defina com suas palavras, o que é conversão eletromecânica da energia?}
\section{Em quais as formas a energia na forma elétrica pode ser convertida?}
\section{Explique com base na ilustração, quais fatores influenciam no valor da tensão induzida no condutor que corta o campo magnético.}

\includegraphics[width=\textwidth]{img1}
\section{Quais são as duas principais partes construtivas de qualquer máquina rotativa?}
\section{Quais as formas mais comuns de se obter o campo magnético em uma máquina CC?}

	\begin{figure}[h!]
		\centering
		\begin{subfigure}[R]{0.3\textwidth}
			\centering
			\includegraphics[width=0.7\textwidth]{img2}
			\caption{Para questão 6}
		\end{subfigure}
		\hspace{2cm}
		\begin{subfigure}[R]{0.3\textwidth}
			\centering
			\includegraphics[width=0.7\textwidth]{img3}
			\caption{Para questão 7}
		\end{subfigure}
	\end{figure}
\section{A figura mostra uma bobina com \(N=200\) espiras. A resistência é \(R=100\Omega\); \(Ag=8cm*12cm; g= 6mm,\) sendo aplicada uma tensão \(V=150V_{DC}\), determine:}
\subsection{Qual a intensidade inicial do campo no entreferro?}
\subsection{A força exercida sobre a parte móvel nesta distância de entreferro}
Após ligar a tensão elétrica, a parte móvel se movimentou, reduzindo o entreferro a \(2mm\).
\subsection{A intensidade do campo e a força depois de ligado o circuito.}
\section{A figura mostra uma bobina com \(N=600\) espiras. A resistência é \(R=100\Omega\); \(Ag=10cm*10cm; g= 15mm,\) sendo aplicada uma tensão \(V=220V_{AC}, 60Hz\). Determine:}
\subsection{A indutância, a reatância indutiva e a impedância da bobina:}
\subsection{Qual a intensidade inicial do campo no entreferro?}
\subsection{Força exercida sobre a parte móvel nesta distância de entreferro.}
\subsection{Após ligar a tensão elétrica, a parte móvel se movimentou, reduzindo o entreferro a \(3mm\).}
\subsection{Intensidade do campo e a força depois de ligado o circuito.}

\end{document}
\date
