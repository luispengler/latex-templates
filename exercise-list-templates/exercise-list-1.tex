\documentclass{article}
\author{Luís Spengler}
\title{Lista de Exercícios - Conversão Eletromecânica}
\date{\today}

\usepackage[utf8]{inputenc}
\usepackage[portuguese]{babel}
\usepackage[margin=1.5cm]{geometry}
\usepackage{amsmath}
\usepackage{physics}
\usepackage{titlesec}
\usepackage{graphicx}
\usepackage{wrapfig}
\usepackage{caption}
\usepackage{subcaption}

\graphicspath{{./docs/}}

\titleformat{\section}
  {\sffamily}{\thesection}{.25em}{- }
\titleformat{\subsection}
  {\sffamily}{\thesubsection}{.01em}{) }
\renewcommand\thesubsection{\alph{subsection}}

\begin{document}
\maketitle
\sffamily

\section{Defina com suas palavras, o que é conversão eletromecânica da energia?}
\textbf{R:} É simplesmente a tranformação de energia elétrica em energia mecânica e vice-versa. Tudo obedece o principio da conservação de energia. Um motor, não sendo 100\% eficiente, só significa que a energia se transformou em outro tipo de energia, ou que ela não foi transmitida de um jeito 100\% efetivo.

\section{Em quais as formas a energia na forma elétrica pode ser convertida?}
\textbf{R:} Em dezenas de maneiras diferentes, mas no nosso dia-a-dia utilizamos de apenas algumas delas. Destas temos as lâmpadas (energia elétrica para luminosa), os autofalantes (energia elétrica para sonora), aquecedores (energia elétrica para energia térmica), ventiladores (energia elétrica para energia mecânica), baterias (energia elétrica para energia potencial química).

\section{Explique com base na ilustração, quais fatores influenciam no valor da tensão induzida no condutor que corta o campo magnético.}
\includegraphics[width=\textwidth]{img1}
\textbf{R:} Velocidade, intensidade do campo magnético e o número de espiras. Quando alguma dessas variáveis é aumentada, a força eletromagnética induzida também aumenta.

\section{Quais são as duas principais partes construtivas de qualquer máquina rotativa?}
\textbf{R:} O rotor (parte móvel) e o estator (parte física).

\section{Quais as formas mais comuns de se obter o campo magnético em uma máquina CC?}
\textbf{R:} Usando um imã permanente ou um eletroimã.
	\begin{figure}[h!]
		\centering
		\begin{subfigure}[R]{0.3\textwidth}
			\centering
			\includegraphics[width=0.7\textwidth]{img2}
			\caption{Para questão 6}
		\end{subfigure}
		\hspace{2cm}
		\begin{subfigure}[R]{0.3\textwidth}
			\centering
			\includegraphics[width=0.7\textwidth]{img3}
			\caption{Para questão 7}
		\end{subfigure}
	\end{figure}

\section{A figura mostra uma bobina com \(N=200\) espiras. A resistência é \(R=100\Omega\); \(Ag=8cm*12cm; g= 6mm,\) sendo aplicada uma tensão \(V=150V_{DC}\), determine:}

\subsection{Qual a intensidade inicial do campo no entreferro?}
\textbf{R:}
\[I=\frac{V}{R} = \frac{150}{100} = 1,5A\]\
\[B_{g}=\frac{\mu_{0}.N.I}{2g} = \frac{4\pi.10^{-7}.200.1,5}{2.0,006} = \frac{\pi}{100} = 0,031415T\]

\subsection{A força exercida sobre a parte móvel nesta distância de entreferro}
\textbf{R:}
\[A_{g}=8cm*12cm = 96.10^{-4}m^2\]\
\[F_{mag} = -\frac{B_{g}^{2}}{\mu_{0}}.A_{g} = -\frac{0,031415^2}{4\pi.10^{-7}}.96.10^{-4} = - 7,53N\]\\
Após ligar a tensão elétrica, a parte móvel se movimentou, reduzindo o entreferro a \(2mm\).

\subsection{A intensidade do campo e a força depois de ligado o circuito.}
\textbf{R:}
\[B_{g}=\frac{\mu_{0}.N.I}{2g} = \frac{4\pi.10^{-7}.200.1,5}{2.0,002} = 0,094247T\]\
\[F_{mag} = -\frac{B_{g}^{2}}{\mu_{0}}.A_{g} = -\frac{0,094247^2}{4\pi.10^{-7}}.96.10^{-4} = -67,8572N \]

\section{A figura mostra uma bobina com \(N=600\) espiras. A resistência é \(R=100\Omega\); \(Ag=10cm*10cm; g= 15mm,\) sendo aplicada uma tensão \(V=220V_{AC}, 60Hz\). Determine:}

\subsection{A indutância, a reatância indutiva e a impedância da bobina:}
\textbf{R:}
\[A=10cm*10cm = 100^{-4}m^2\]\
Indutância: \[L=\frac{\mu_{0}.N^2.A}{2.l_g} = \frac{4\pi.10^{-7}.600^2.100.10^{-4}}{2.0,015} = 0,1507 = 15,07mH\]\
Reatância Indutiva: \[X_l=2\pi.f.L = 2\pi.60.15,07.10^{-3} = 5,68\Omega\]\
Impedância: \[Z = R + jX_l = 100 + j5,68\Omega = 100,1611\Omega\] 

\subsection{Qual a intensidade inicial do campo no entreferro?}
\textbf{R:}
\[I=\frac{V}{|Z|} = \frac{220}{100,1611} = 2,2A\]\
\[B_{g}=\frac{\mu_{0}.N.I}{2g} = \frac{4\pi.10^{-7}.600.2,2}{2.0,015} = 0,055T\]

\subsection{Força exercida sobre a parte móvel nesta distância de entreferro.}
\textbf{R:}
\[F_{mag} = -\frac{B_{g}^{2}}{\mu_{0}}.A = -\frac{0,055^2}{4\pi.10^{-7}}.100.10^{-4} = - 24,32N\]\\
Após ligar a tensão elétrica, a parte móvel se movimentou, reduzindo o entreferro a \(3mm\).
\subsection{Intensidade do campo e a força depois de ligado o circuito.}
\textbf{R:}
\[A=10cm*10cm = 100^{-4}m^2\]\
\[L=\frac{\mu_{0}.N^2.A}{2.l_g} = \frac{4\pi.10^{-7}.600^2.100.10^{-4}}{2.0,003} = 0,75398H\]\
\[X_l=2\pi.f.L = 2\pi.60.0,75398 = 284,244\Omega\]\
\[Z = R + jX_l = 100 + j284,244\Omega = 301,3215\Omega\]\
\[I=\frac{V}{|Z|} = \frac{220}{301,3215} = 0,7301A\]\
\[B_{g}=\frac{\mu_{0}.N.I}{2g} = \frac{4\pi.10^{-7}.600.0,7301}{2.0,003} = 0,09174T\]\
\[F_{mag} = -\frac{B_{g}^{2}}{\mu_{0}}.A_{g} = -\frac{0,09174^2}{4\pi.10^{-7}}.100.10^{-4} = -66,97N \]

\end{document}
